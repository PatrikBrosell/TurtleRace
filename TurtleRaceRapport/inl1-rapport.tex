% !TEX encoding = UTF-8 Unicode
\documentclass[a4paper]{article}

\usepackage[T1]{fontenc}     % För svenska bokstäver
\usepackage[utf8]{inputenc}  % Teckenkodning UTF8
\usepackage[swedish]{babel}  % För svensk avstavning och svenska
                             % rubriker (t ex "Innehållsförteckning")
\usepackage{fancyvrb}        % För programlistor med tabulatorer
\fvset{tabsize=4}            % Tabulatorpositioner
\fvset{fontsize=\small}      % Lagom storlek för programlistor

\title{TurtleRace \\
	Inlämningsuppgift 1, Programmeringsteknik}
\author{Patrik Brosell, D12 (dat12pbr@student.lu.se)\\
Fredrik Danebjer, D12 (dat12fda@student.lu.se}
%\date{2 november 2012}        % Blir dagens datum om det utelämnas

% *** Tillägg för denna rapport. ***
% Paket:
\usepackage{graphicx}         % För att inkludera bilder.

% Kommandon i denna rapport
\newcommand{\code}[1]{\texttt{#1}} % För programkod i text.
% *** Slut på tillägg för denna rapport. ***

\begin{document}              % Början på dokumentet

\maketitle                    % Skriver ut rubriken som vi
                              % definierade ovan med \title, \author
                              % och eventuellt \date

\section{Bakgrund}
En kapplöpningstävling mellan \texttt{n} antal deltagare äger rum på en kapplöpningsbana ska simuleras. Banan har ett fixt utseende, till exempel en bestämd längd och bredd. Flera lopp kan avgöras på samma bana.

Skiss över anläggningen:

\begin{center}
\includegraphics[scale=0.2]{turtlerace.pdf}
\end{center}

Ett program som simulerar ett lopp mellan \texttt{n} sköldpaddor på en kapplöpningsbana skall skrivas. Kapplöpningsbanan består av \texttt{n} antal banor, en för varje sköldpadda. Alla banor har en gemensam
 start- samt mållinje. I figuren syns startlinje, mållinje och 5 tävlande sköldpaddor.


\section{Modell}
Modellen av kapplöpningsbanan innehåller följande klasser:

\begin{tabular}{lp{8cm}}
\code{TurtleRace} & Innehåller \code{main}-metoden som skapar sköldpaddorna. \\
\code{RaceTrack} & Skapar en kapplöpningsbana. \\
\code{RacingEvent} & Skapar ett lopp mellan sköldpaddorna på kapplöpningsbanan \code{track}. \\
\code{Turtle} & Beskriver en sköldpadda som kan springa i ett \code{SimpleWindow}-fönster. \\
\end{tabular}

\vspace{\baselineskip}
Simuleringen implementeras genom att programmet under hela simuleringstiden varje minut undersöker vad som ska ske. Följande händelser kan inträffa:

\begin{itemize}
\item En bil anländer till systemet. Detta hanteras av klassen \code{Entry}, som varje minut drar ett Poisson-fördelat slumptal som anger antalet bilar som anländer.
\item En bil börjar tvättas (flyttas från kön till en tvättstation). Detta hanteras av klassen \code{CarWash}.
\item En bil är färdigtvättad (flyttas från en tvättstation till utfarten). Också detta hanteras av klassen \code{CarWash}.
\item En bil lämnar systemet. Detta hanteras av klassen \code{Exit}.
\end{itemize}


De nämnda ''aktiva'' klasserna har alla en metod \code{tick()} som anropas varje minut. Även klassen \code{Clock}, som håller reda på tiden i systemet, har en sådan metod.

Bilarna (\code{Car}-objekten) är helt passiva och skickas bara runt i systemet av de aktiva klasserna. Klasserna \code{Queue} och \code{WashingStation} är också passiva. Deras uppgift är att hålla reda på \code{Car}-objekt under tiden bilarna står i kö eller tvättas. 


\section{Brister och kommentarer}
Eftersom utskriften för varje bil sker när bilarna lämnar systemet kan det inträffa att bilarna inte skrivs ut i nummerordning. Det kan till exempel se ut på följande vis (bilnummer, ankomst\-minut, avgångsminut, tvättstation):

\begin{Verbatim}
     4   5  24   2
     5   7  25   1
     7  21  33   1
     6  17  34   2
\end{Verbatim}

Uppgiften har lösts enligt specifikationen med två tvättstationer i biltvätten. Det innebär dock inga större svårigheter att ändra lösningen så att den klarar av ett godtyckligt antal tvätt\-stationer. Tiden som simuleringen ska pågå (i uppgiften var det 60 minuter) läses in när programmet startas, liksom medelvärdet för antalet bilar som ankommer per minut.


\section{Programlistor}
Klasserna finns i filer med samma namn som klasserna, till exempel finns klassen \code{SimWash} i filen \code{SimWash.java}. Samtliga filer finns i samma katalog som huvudprogrammet.

\subsection{\code{SimWash}}

% *** Observera: här ligger java-filerna i samma katalog som 
% *** LaTeX-filen. Det är inte nödvändigt; man kan skriva ett
% *** absolut filnamn (med sökvägen till java-filerna i ert
% *** Eclipse-projekt).
\VerbatimInput{SimWash.java}

\subsection{\code{CarWash}}

\VerbatimInput{CarWash.java}

\ldots\ osv, liknande programlistor för samtliga klasser.

\end{document}                  % Slut på dokumentet
